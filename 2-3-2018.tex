%%% Research Diary - Entry
%%% Template by Mikhail Klassen, April 2013
%%% 
\documentclass[11pt,letterpaper]{article}

\newcommand{\workingDate}{\textsc{2018 $|$ February $|$ 03}}
\newcommand{\userName}{Yiluo Li}
\newcommand{\institution}{UC Santa Barbara}
\usepackage{researchdiary_png}
% To add your univeristy logo to the upper right, simply
% upload a file named "logo.png" using the files menu above.

\begin{document}
% \univlogo

\title{Research Diary}

{\Huge February 3}\\[5mm]

\section*{Overview of the tutorial}

\subsection*{Intro to Signal Processing}
This turns out to be a vast field of study in which the most important one here is digital time series data, including the following topics that I absolutely need to learn about:
\begin{enumerate}
    \item Spectral density
    \item Spectrograms
    \item Digital filtering
    \item Whitening
    \item Audio manipulation
\end{enumerate}

I am currently learning from Coursera DSP course and textbook \textit{The Scientist and Engineer's Guide to Digital Signal Processing}.

\section{Time Series Data Analysis}
As opposed to cross-sectional data analysis, which studies several independent time series at a single time point, time series data analysis studies one or multiple dependent time series. \par
Analog data are the ones with continuous domain, uncountable set, while digital data have discrete domain, countable set, which only display finite amount of data points over a finite time period. \par
Something note worthy is the mathematical methods used in these analysis:
\begin{enumerate}
    \item Discrete Fourier Transform
    \item Inverse DFT
    \item Fast Fourier Transform
    \item z-transform
\end{enumerate}

\textit{Seems like I also need some knowledge with Hilbert Space.}
\end{document}